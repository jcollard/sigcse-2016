% This is "sig-alternate.tex" V2.0 May 2012
% This file should be compiled with V2.5 of "sig-alternate.cls" May 2012
%
% This example file demonstrates the use of the 'sig-alternate.cls'
% V2.5 LaTeX2e document class file. It is for those submitting
% articles to ACM Conference Proceedings WHO DO NOT WISH TO
% STRICTLY ADHERE TO THE SIGS (PUBS-BOARD-ENDORSED) STYLE.
% The 'sig-alternate.cls' file will produce a similar-looking,
% albeit, 'tighter' paper resulting in, invariably, fewer pages.
%
% ----------------------------------------------------------------------------------------------------------------
% This .tex file (and associated .cls V2.5) produces:
%       1) The Permission Statement
%       2) The Conference (location) Info information
%       3) The Copyright Line with ACM data
%       4) NO page numbers
%
% as against the acm_proc_article-sp.cls file which
% DOES NOT produce 1) thru' 3) above.
%
% Using 'sig-alternate.cls' you have control, however, from within
% the source .tex file, over both the CopyrightYear
% (defaulted to 200X) and the ACM Copyright Data
% (defaulted to X-XXXXX-XX-X/XX/XX).
% e.g.
% \CopyrightYear{2007} will cause 2007 to appear in the copyright line.
% \crdata{0-12345-67-8/90/12} will cause 0-12345-67-8/90/12 to appear in the copyright line.
%
% ---------------------------------------------------------------------------------------------------------------
% This .tex source is an example which *does* use
% the .bib file (from which the .bbl file % is produced).
% REMEMBER HOWEVER: After having produced the .bbl file,
% and prior to final submission, you *NEED* to 'insert'
% your .bbl file into your source .tex file so as to provide
% ONE 'self-contained' source file.
%
% ================= IF YOU HAVE QUESTIONS =======================
% Questions regarding the SIGS styles, SIGS policies and
% procedures, Conferences etc. should be sent to
% Adrienne Griscti (griscti@acm.org)
%
% Technical questions _only_ to
% Gerald Murray (murray@hq.acm.org)
% ===============================================================
%
% For tracking purposes - this is V2.0 - May 2012

\documentclass{sig-alternate}

\begin{document}
%
% --- Author Metadata here ---
\conferenceinfo{WOODSTOCK}{'97 El Paso, Texas USA}
%\CopyrightYear{2007} % Allows default copyright year (20XX) to be over-ridden - IF NEED BE.
%\crdata{0-12345-67-8/90/01}  % Allows default copyright data (0-89791-88-6/97/05) to be over-ridden - IF NEED BE.
% --- End of Author Metadata ---

\title{Captain Teach 2.0}

\numberofauthors{3} %  in this sample file, there are a *total*
% of EIGHT authors. SIX appear on the 'first-page' (for formatting
% reasons) and the remaining two appear in the \additionalauthors section.
%
\author{
% You can go ahead and credit any number of authors here,
% e.g. one 'row of three' or two rows (consisting of one row of three
% and a second row of one, two or three).
%
% The command \alignauthor (no curly braces needed) should
% precede each author name, affiliation/snail-mail address and
% e-mail address. Additionally, tag each line of
% affiliation/address with \affaddr, and tag the
% e-mail address with \email.
%
% 1st. author
\alignauthor
Joseph Collard\\
       \affaddr{University of Massachusetts Amherst}\\
       \email{jcollard@cs.umass.edu}
% 2nd. author
\alignauthor
2nd Author
       \affaddr{Some University}\\
       \email{fake@email.com}
% 3rd. author
\alignauthor 
3rd Author
       \affaddr{Some University}\\
       \email{fake@email.com}
}

\maketitle
\begin{abstract}
Previously an in-flow peer review tool named Captain Teach was created for use 
in two courses at Brown. Captain Teach focused on creating an environment for
performing peer review on assignments that are still in progress.  
The tool was dependent on the Pyret programming language and was designed for 
a very specific workflow. This paper describes Captain Teach 2.0, the second 
iteration of this in-flow peer review tool which is a language independent tool 
that utilizes YAML assignment descriptions which can be used to configure various
linear assignment workflows. 
\end{abstract}

% A category with the (minimum) three required fields
\category{K.3.2}{Computers and Education}{Computer and Information Science Education}

\keywords{Peer-review}

\section{Introduction}
To be completed


\section{Captain Teach Workflows}
Captain Teach is designed to be used on assignments which have multiple
deliverables. Each deliverable is decomposed into a submission step typically
followed by a student reviewing one or more other submissions before being
allowed to submit their next deliverable. This gives the student an
opportunity to see how others have approached the same problem and possibly
catching mistakes and misunderstandings prior to submitting their completed
assignment.

\subsection{Authoring an Assignment Description}
To accommodate various assignment workflows that an instructor might want to use
in their course, Captain Teach provides a YAML based assignment configuration 
format for describing an assignment. This description specifies a sequential 
list of submission steps. For each step an instructor may specify that a 
student must complete one or more reviews before submitting the next deliverable.
These reviews may be of others students deliverables for the associated step or 
an instructor provided deliverable. An instructor provided review might be a
known good or bad solution of the current step or simply a mid-assignment 
survey for students to complete. Finally, the instructor is able to provide
customized rubrics for each of the reviews. (TODO: Add in figure showing authoring screen / resulting rubric)


\subsection{Student Dashboard}
When a student accesses their course page, they are presented with a list of
open and closed assignments. A closed assignment is one which a student
is no longer allowed to submit deliverables.
Accessing an assignment page brings the student,
to a dashboard showing all of the work they have submitted, any reviews
they have completed and any reviews which they have received on their own work.
In addition, a student accessing an open assignment is presented with the next
action they must complete to make progress on the assignment: submit a 
deliverable or complete reviews. If a students has finished all steps on an
assignment, they are told they have nothing left to do on this assignment. 
TODO: Assignment dashboard figure)

After submitting a deliverable for a step, students will be given any
reviews that the instructor has specified. The assigned reviews are anonymous
and are only distinguishable on the dashboard by a number specifying the order
they were assigned to the student. Unless specifically noted on the review
rubric or provided deliverable an instructor provided deliverable looks the
same as any students deliverable. Reviewers are free to complete their reviews
in any order and are able to view all of them immediately.

Upon opening a review, students are presented with a file browser like
interface as well as the instructor provided rubric. Selecting a source code
file from the browser will open it in a Code Mirror editor which provides
syntax highlighting for many programming languages. The editor is set to read
only mode. However, clicking on a line of code in the editor causes a text
area to appear below the line where the student may leave in-line feedback for
the reviewee. After completing a review, the reviewee receives a notification 
via email with a link to the assignment dashboard where they may now access 
the completed review. (TODO: Figure of Review Screen)

\subsection{Instructor Dashboard}
Each assignment comes with a dashboard the instructor may use to 
open and close assignments, get quick overview of the progress students 
have made, manage individual submissions, and make changes to the assignment
description.

Accessing an assignment dashboard presents you with a high level overview of
each of the steps. This page shows the number of students who have uploaded
an assignment, published their work for review, and completed reviews.
A link for each of these subsections provides the instructor with a more
detailed table of when the deliverable was uploaded, who has been assigned
to review the deliverable, when the deliverable was reviewed, when a completed
review was first accessed by a reviewee and if the reviewee has flagged the
review for the instructor. Each of these items has a link on them which will
open a page showing the submission or review so the instructor can easily
see the same screen as the student. In addition to this, there are buttons
next to each submission / review which allows the instructor to re-open 
so a student may make revisions and resubmit.  
(TODO: Insert figure of dashboard).

\section{Conclusions}
To be completed


\section{Related Work}
To be completed


\section{Acknowledgments}
We would like to thank the staff and students of Brown's CS173, CS019, 
Worcester Polytechnic Institute's CS4536, and the University of Massachusetts
CS220 for all of their help in testing and creating Captain Teach.



%
% The following two commands are all you need in the
% initial runs of your .tex file to
% produce the bibliography for the citations in your pap er.
\bibliographystyle{abbrv}
\bibliography{cap-bib}  % sigproc.bib is the name of the Bibliography in this case
% You must have a proper ".bib" file
%  and remember to run:
% latex bibtex latex latex
% to resolve all references
%
% ACM needs 'a single self-contained file'!
%

\end{document}
